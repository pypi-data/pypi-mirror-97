\documentclass[fleqn,oneside,12pt]{article}
\date{}
\newcommand{\up}[1]{a_{#1}^{\dagger}}
\newcommand{\down}[1]{a_{#1}}
\newcommand{\ud}{\mathrm{d}}
\newcommand{\EE}{\mathrm{E}}
\newcommand{\diff}[2]{\frac{\ud {#1}}{\ud {#2}}}
\newcommand{\pdiff}[2]{\frac{\partial #1}{\partial #2}}
\newcommand{\fl}{\mathrm{fl}}
\newcommand{\sqrtfrac}[2]{\sqrt{\frac{#1}{#2}}}
\newcommand{\ui}{\mathbf{i}}
\newcommand{\uj}{\mathbf{j}}
\newcommand{\ua}{\mathbf{a}}
\newcommand{\ub}{\mathbf{b}}
\newcommand{\bi}{\bar{i}}
\newcommand{\bj}{\bar{j}}
\newcommand{\ba}{\bar{a}}
\newcommand{\bb}{\bar{b}}
\newcommand{\tb}{\phantom{}^{Q} \hat b}
\newcommand{\erf}{\mathrm{erf}}
\newcommand{\erfc}{\mathrm{erfc}}
\setlength{\parskip}{1ex plus 0.5ex minus 0.2ex}
\setlength{\parindent}{0pt}
\usepackage[top=1in, bottom=1in, left=1in, right=1in]{geometry}
\usepackage{amsmath}
\usepackage[super]{cite}
\usepackage{amssymb}
\usepackage[usenames,dvipsnames]{color}
\usepackage{verbatim}
\usepackage{fancyhdr}
\usepackage{graphicx}
\pagestyle{fancyplain}
\begin{document}
\lhead{Ultrafast X-Ray Scattering}
\rhead{Rob Parrish and Todd Mart\'inez}

\section{Development}

\textbf{Geometry:} Consider an ultrafast time-resolved optical pump and elastic
X-ray scattering probe experiment in a perpendicular arrangement, as shown in
Figure \ref{fig:schem}. Initially, the samples are all oriented randomly, but
the optical pump preferentially excites chromophores with transition dipoles
$\mu$ oriented along $\hat z$, with probability $P (\hat \mu) \equiv \hat
\mu_z^2 = \cos^2 (\zeta_{\mu})$, where $\zeta_{\mu}$ is the polar angle from
$\hat z$.  This anisotropic absorption will induce anisotropy in the subsequent
X-ray scattering signal.

In the independent atom model (IAM), the X-ray scattering signal is,
\[
I (\vec q, t)
\equiv
\frac{1}{N_I}
\sum_{I}
\left |
\sum_{A}
f_{A} (q)
e^{i \vec q \cdot \vec r_A^I (t)}
\right |^2
\]
Here $\vec r_A^I (t)$ are the nuclear coordinates of atom $A$ as a function of
delay time $t$, for trajectory $I$ [from molecular dynamics trajectories on the
relevant electronic state(s)]). By averaging over many trajectories, and by
weighting to account for the $\cos^2 (\zeta_{\mu})$ anisotropy of the initial
conditions, the time-resolved X-ray scattering signal can be obtained on a
one-to-one footing with the experiment. The $f_A (q)$ are atom- and
X-ray-beam-specific scattering cross sections, and have been tabulated online in
a standard Gaussian form.

The X-ray scattering signal is best written in spherical coordinates as $I(\vec
q, t) \equiv I(q, \zeta, \phi, t)$. Here $q$ is the scattering amplitude,
$\zeta$ is the polar angle from $\hat z$, and $\phi$ is the azimuthal angle in
$\hat x$-$\hat y$. Note that we expect the signal to be isotropic in $\phi$, but
not in $\zeta$. 

\begin{figure}[h!]
\begin{center}
\includegraphics[width=11cm]{figures/schem}
\end{center}
\caption{An illustration of ultrafast X-ray diffraction. The X-ray probe arrives
a delay of $t$ with respect to the optical pump. In the perpendicular
experimental arrangement (shown here), the polarization vector $\vec E \propto
\hat z$ of the optical pump pulse is perpendicular to the incident wave vector
$\vec k_0 \equiv k \hat y$ of the X-ray probe pulse. The incident X-ray photons
scatter at angle $\theta$ with respect to $\vec k_0$, and azimuthal angle $\eta$
with respect to $\vec E$. These two angles fully describe the scattering in the
elastic regime, as $|\vec k_0|$ = $|\vec k_s|$. The scattering vector $\vec q$
is the difference between the incident and scattered wave vectors $\vec q \equiv
\vec k_0 - \vec k_s$. The amplitude of the scattering vector is $q \equiv |\vec
q| = 2 k \sin \theta / 2$. Figure adapted to our notation from Kent Wilson's
excellent 1998 ultrafast scattering paper in JPCA [Jianshu Cao and Kent R.
Wilson, \emph{J. Phys. Chem. A} \textbf{102}, 9523 (1998)].}
\label{fig:schem}
\end{figure}

\textbf{Spherical Harmonic Transformation:} A useful transformation of the X-ray
scattering signal involves the (complete) projection onto spherical harmonics,
\[
I (\vec q, t)
=
\sum_{lm}
Y_{lm} (\zeta, \phi)
I_{lm} (q, t)
\]
where,
\[
I_{lm} (q, t)
\equiv
\int_{\Omega}
\mathrm{d} \Omega \
Y_{lm}^{*} (\Omega)
I (q, \Omega, t)
\]
The solid angle $\Omega$ is shorthand for the set of $<\zeta, \phi>$
coordinates.  Here the spherical harmonics are defined to be fully orthonormal
on the unit sphere,
\[
\int_{\Omega}
\mathrm{d} \Omega \
Y_{lm}^{*} (\Omega)
Y_{l'm'} (\Omega)
=
\delta_{ll'}
\delta_{mm'}
\]
Due to parity, the odd-$l$ spherical harmonic contributions must be zero. Due to
isotropy in $\phi$, the $m \neq 0$ spherical harmonic contributions must be
zero. Thus, the only surviving contributions are the even zonal spherical
harmonics $Y_{l0}$ where $l$ is even.

The normalized zonal spherical harmonics are,
\[
Y_{l0} (\zeta, \phi)
\equiv
N_{l} 
P_{l} (\cos \zeta)
\]
The Legendre polynomials are,
\[
P_{0} (x) = 1
\]
\[
P_{1} (x) = x
\]
\[
P_{l+1} (x)
=
\frac{1}{l+1}
\left [
(2l + 1) x P_{l} (x)
-
l P_{l-1} (x)
\right ]
\]
The normalization factors are,
\[
N_{l}
\equiv
\sqrt{\frac{2l + 1}{4 \pi}}
\]

\textbf{Tasking:} For a set of trajectories $r_{A}^{I} (t)$ aligned so that
$\vec \mu (t = 0) = \mu \hat z$, we must compute,
\[
I_{l0} (q, t)
\equiv
\frac{1}{N_{I}}
\sum_{I}
\int_{\mathrm{SO}(3)}
\mathrm{d} \hat R \
\hat R_{z}^{2}
\int_{\Omega}
\mathrm{d} \Omega \
Y_{l0}^{*} (\Omega)
\left |
\sum_{A}
f_{A} (q)
e^{i \vec q \cdot [\hat R \vec r_A^I (t)]}
\right |^2
\]
Colloquially, this says ``rotate each trajectory through the full set of
rotations $\hat R$, and weight these by the square of the projection of the
rotation on $\hat z$ ($\hat R_z^2 \Leftrightarrow \cos^2 [\zeta (\hat R)]$) to
account for the excitational anisotropy. For each trajectory and rotation,
compute the diffraction signal in $\vec q$, and project onto spherical harmonic
contributions $I_{l0}^{I, \hat R} (q, t)$. Average these over all rotations and
all trajectories.''

A key observation is that one can apply the rotations to the diffraction
patterns, rather than to the trajectories, which allows for interchange of
summation,
\[
\left |
\sum_{A}
f_{A} (q)
e^{i \vec q \cdot [\hat R \vec r_A^I (t)]}
\right |^2
=
\left |
\sum_{A}
f_{A} (q)
e^{i [\hat R^\dagger \vec q] \cdot \vec r_A^I (t)}
\right |^2
\]
This amounts to switching working variable from $\vec q = <q, \Omega>$ to $\vec
q' = <q, \hat R^\dagger \Omega>$ and leaving the trajectories in $\hat z$ frame.
The observable may now be written as,
\[
I_{l0} (q, t)
\equiv
\int_{\Omega}
\mathrm{d} \Omega \
\int_{\mathrm{SO}(3)}
\mathrm{d} \hat R \
\hat R_{z}^2
Y_{l0}^{*} (\hat R^\dagger \Omega)
\underbrace{
\frac{1}{N_{I}}
\sum_{I}
\left |
\sum_{A}
f_{A} (q)
e^{i \vec q' \cdot \vec r_A^I (t)}
\right |^2
}_{
I^{\hat z} (\vec q', t)
}
\]

\textbf{Computational Procedure:} 

\begin{enumerate}
\item Compute $N_I$ trajectories $r_{A}^{I} (t)$ and align these so that 
$\vec \mu (t = 0) = \mu \hat z$.
\item Lay out a regular grid $q_P$ and a Lebedev grid $<\Omega_Q, w_Q>$ to form
a 3D spherical grid $\vec q_R \equiv q_{P} \otimes \Omega_{Q}$.
\item Evaluate,
\[
I^{\hat z} (\vec q_R, t)
\equiv
\frac{1}{N_I}
\sum_{I}
\left |
\sum_{A}
f_{A} (q_R)
e^{i \vec q_R \cdot r_A (t)}
\right |^2
\]
\item Generate a covering sequence of rotation matrices and affiliated weights
$<\hat R_{M}, w_{M}>$ to integrate SO$(3)$. A particularly nice way to do this
is to exploit the ``rotate around $\hat z$ by angle $\omega$, then rotate down
to a uniform spherical angle $\Omega$'' method for generating uniform random
rotation matrices, but adapting this to the appropriate regular quadrature grids
for $\omega$ and $\Omega$ (Fourier and Lebedev).  That is, lay out uniform
Fourier grid $\omega_{T} \equiv 2 \pi T / (N_{T} + 1)$ and $w_T = 1 / N_{T}$ and
a Lebedev grid $<\Omega_{U}, w_{U}>$. Then form the direct product grid
$\Phi_{M} \equiv \omega_{T} \otimes \Omega_{U}$ and $w_{M} \equiv (w_{T} \otimes
w_{U}) / 4 \pi$.  For each $M$ point, start with the identity matrix, rotate
about $\hat z$ by $\omega_T$ (a simple Given's rotation), and then rotate
directly down from $\hat z$ to $\Omega_M$ (a Householder reflection + parity
inversion) to obtain the rotation matrix $\hat R_M$.

\item Form the observable,
\[
I_{l0} (q_{P}, t)
=
\sum_{Q}
w_{Q}
\sum_{M}
w_{M}
(\hat R_{M}^{z})^2
Y_{l0} (\hat R_{M}^{\dagger} \Omega_{Q})
I^{\hat z}
(q_{P}, \Omega_{Q}, t)
\]

\end{enumerate}

\textbf{X-Ray Scattering vs. Electron Diffraction:} For ultrafast x-ray
scattering, the form factors $f_{A} (q)$ are the Fourier transform of the
electronic density of a spherical atom (possibly selected to reflect the density
of the local chemical environment in the true molecule),
\[
f_{A} (q)
\equiv
\int_{\mathbb{R}^3}
\mathrm{d} r_1
\
\rho_{A}^{\mathrm{e}} (\vec r_{1})
e^{i \vec q \cdot \vec r_1}
\]

These form factors are tabulated for many common atoms at
\begin{verbatim}
http://lampx.tugraz.at/~hadley/ss1/crystaldiffraction/atomicformfactors/formfactors.php
\end{verbatim}
(accessed 01/18/2018). The specific parametrization is,
\[
f_{A} (q)
\equiv
\sum_{i=1}^{4}
a_{i}
\exp
\left (
-b_{i}
\left (
\frac{q}{4\pi}
\right )^2
\right )
+
c
\]
With 9 parameters $a_{1-4}$ $b_{1-4}$, and $c$ per atom type. Units are
$\AA{}^{-1}$.

To switch to ultrafast electron diffraction, one must also account for
scattering off of the nucleus and for a modified Jacobian element,
\[
f_{A}^{\mathrm{UED}}
(q)
\equiv
\frac{1}{q^2}
\left [
Z_{A}
-
f_{A}^{\mathrm{XRAY}}
(q)
\right ]
\]
See M. Ben-Nunn, J. Cao, and K. Wilson, JPCA, 101, 8744 (1997) for details.

\end{document}
